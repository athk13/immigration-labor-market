\begin{frame}
    \frametitle{Conclusions}
        \begin{itemize}
            \item This study examined the impact of large immigrant inflows on native labor market outcomes across U.S. regions using 2SLS regression, comparing two instrumental variables: the Standard Card Instrument and Predicted Immigrant Growth Rate.
            \item \textbf{Instrumental Relevance:} The Predicted Immigrant Growth Rate proved more robust and statistically significant, meeting the Partial F-statistic threshold of 10 with all controls included.
            \item \textbf{Labor Market Outcomes:}
            \begin{itemize}
                \item Immigrant inflows had no significant effect on native wage growth, countering concerns of wage depression.
                \item A positive and significant relationship was found between immigrant inflows and native unemployment, suggesting regional labor market frictions.
                \item Immigrant inflows contributed to modest increases in the NILF (not in labor force) rate, indicating potential workforce shifts.
            \end{itemize}
            \item Comparing Commuting Zones (CZs) with higher and lower exposure to immigrant inflows, more exposed CZs experienced a 4.1% decline in wages and a 3.8 percentage point rise in unemployment, emphasizing regional disparities.
            \item These findings underscore the importance of robust instruments in immigration research and the need for policies that address labor market adjustments and reduce regional inequalities.
        \end{itemize}
\end{frame}

