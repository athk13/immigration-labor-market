\begin{frame}
    \frametitle{Conclusions}
    \begin{itemize}
        \item This study examined the impact of large immigrant inflows on native labor market outcomes across U.S. regions using 2SLS regression and two instrumental variables: the Standard Card Instrument and Predicted Immigrant Growth Rate.
        \item \textbf{Instrumental Relevance:} The Predicted Immigrant Growth Rate outperformed the Standard Card Instrument in robustness and statistical significance, meeting the Partial F-statistic threshold of 10 with all controls included.
        \item \textbf{Labor Market Outcomes:}
        \begin{itemize}
            \item Immigrant inflows had no significant effect on native wage growth, alleviating concerns about wage depression.
            \item A positive and significant relationship was observed between immigrant inflows and native unemployment, suggesting potential labor market frictions in specific regions.
            \item No consistent impact was found on labor force participation (LFP).
        \end{itemize}
        \item The findings highlight the importance of robust instruments in immigration research and suggest that while wages remain unaffected, rising unemployment rates in certain regions require further investigation.
        \item Reducing regional disparities and addressing labor market adjustments caused by immigration warranst further discussion.
    \end{itemize}
\end{frame}

